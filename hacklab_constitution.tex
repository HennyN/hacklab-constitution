% Hacklab constitution
\documentclass{article}
\usepackage{palatino}
\title{Constitution of the Edinburgh Hacklab}
\date{26th April 2011}
\renewcommand{\labelenumi}{Article \arabic{enumi}:}
\renewcommand{\labelenumii}{\arabic{enumi}.\arabic{enumii}}
\renewcommand{\labelenumiii}{\arabic{enumi}.\arabic{enumii}.\arabic{enumiii}}
\begin{document}
\maketitle
\begin{enumerate}
  \item Name
    \begin{enumerate}
      \item The name of the organisation shall be ``Edinburgh
        Hacklab'', hereafter referred to as ``the Hacklab''.
      \item The Hacklab shall hold its regular activities in Edinburgh.
    \end{enumerate} % Name
  \item Purpose
    \begin{enumerate}
    \item The main purpose of the Hacklab shall be to provide a shared
      physical workspace, tools, storage and other resources for its
      members.
      \item In addition to this main purpose, the Hacklab shall also aspire to:
        \begin{itemize}
        \item promote and and encourage technical, scientific and
          artistic skills and innovation through individual projects,
          collaboration and education.
        \item provide events to allow the wider community to meet and
          socialise together. 
          \item promote and support the use and development of free
            and open technologies, standards, ideas, hardware and
            software for the benefit of all.
          \item work with other bodies with similar or complementary
            objectives.
        \end{itemize}
    \end{enumerate} % Purpose
  \item Organisation
    \begin{enumerate}
    \item The Hacklab shall be run by its members. A limited set of
      finance, membership and management tasks may be performed by a
      committee on behalf of the members.
    \item All decisions of the Hacklab shall be made by consensus, or
      if consensus cannot be reached then at a meeting by majority
      vote. In this constitution, consensus means that all full
      members have been given the opportunity to object to a decision
      and no reasonable objections have been received within one week.
    \item All members and guests shall refrain from actions or
      behaviour which would have a negative effect on the Hacklab.
    \item The Hacklab shall not be held responsible or liable for
      any actions or behaviour of individuals or groups, whether
      members or guests.
    \item This constitution may only be amended at a general meeting.
    \item In addition to this constitution, the Hacklab shall maintain
      a set of Operating Procedures.
    \item All members and guests shall be bound by both this
      constitution and the current Operating Procedures.
    \end{enumerate} % Organisation
  \item Membership
    \begin{enumerate}
    \item A full member shall have the right to keys and be permitted
      to attend any physical workspaces at any time, to have a single
      vote, to make use of resources and to accompany guests.
    \item Full members shall be 18 years old or older.
    \item The process for acceptance of new members shall be set out
      in the Operating Procedures.
    \item Membership fees shall be set out in the Operating Procedures. 
    \item The membership fee shall be paid in advance. The form of
      payment shall be monthly by electronic bank transfer or monthly
      by cash.
    \item In extreme circumstances, any member or guest can
      be suspended by unanimous agreement of the committee or following
      consensus decision by the members of the Hacklab. Suspended
      members or guests shall be denied access to any physical
      workspaces until the reason for their suspension has been
      resolved. A General Meeting is to be held if a full member is to
      be removed from the Hacklab.
    \item If a member wishes to leave the Hacklab, written notice
      shall be sent to the mailing list or alternatively may be given
      to a committee member. Continued non-payment of the membership
      fee shall be deemed to be the same as the receipt of written
      notice to leave the Hacklab.
    \item Upon leaving or being removed from the Hacklab, all keys,
      access tokens or Hacklab property must be returned to the Hacklab. 
    \item On the adoption of this constitution all people who have
      paid membership fees prior to its adoption and have agreed to
      the constitution shall be deemed to be full members.
    \end{enumerate} % Membership
  \item Committee
    \begin{enumerate}
    \item The tasks of the committee shall be to maintain the
      membership list, coordinate the financial activities and act as
      signatories on behalf of the Hacklab.
    \item The committee shall be made up of full members and shall consist of:
      \begin{itemize}
      \item One Treasurer
      \item Two ordinary committee members
      \end{itemize}
    \item Committee members shall receive no remuneration for being
      part of the committee, other than for reasonable expenses on
      presentation of a receipt.
    \item A committee member shall not serve for more than 3 consecutive 
      years.
    \item The Treasurer shall maintain the Hacklab's financial
      accounts and membership list.
    \item If a committee member resigns from the committee or leaves
      the Hacklab a general meeting shall be called to elect a
      replacement.
    \item A general meeting shall be called if a committee member is
      to be removed from the committee.
    \item All personal information shall be maintained in strict
      confidence by the committee.
    \end{enumerate} % committee
  \item Finance
    \begin{enumerate}
    \item The Hacklab shall be non-commercial and not-for-profit.
    \item All income shall be used to achieve the Hacklab's objectives.
    \item The primary income source for the organisation shall be
      the membership fee. All other sources shall be secondary income.
    \item The committee, in close consultation with the membership, shall
      coordinate expenditure on the running costs of the Hacklab.
    \item The signature or electronic authorisation of two of the
      committee members shall be required for all cheques or
      financial transactions of the Hacklab.
    \item Accounts shall be reported and audited annually. The
      financial year shall run from 1st April to 31st March.
    \end{enumerate}
    \item Use of Resources
      \begin{enumerate}
      \item The Hacklab shall make no claim and take no responsibility
        for the projects created by members using Hacklab resources.
      \item Use of Hacklab facilities and equipment shall be at
        members' and guests' own risk.
      \end{enumerate}
    \item General Meetings
      \begin{enumerate}
      \item There shall be an Annual General Meeting (AGM) held in
        April each year.
      \item An Extraordinary General Meeting (EGM) shall be convened if:
        \begin{itemize}
        \item The lower of ten percent of full members or ten full
          members request one;
        \item or the committee requests one.
        \end{itemize}
      \item Quorum of a general meeting shall be two committee members
        and the lower of ten percent of full members or ten full
        members.
      \item The agenda of the AGM shall include annual report from the
        Treasurer, election of a new committee, and the setting of the
        membership fee.
      \item The agenda of the EGM shall be restricted to the items
        given in the notice of the EGM.
      \item Notice of all general meetings and the agenda of the
        meeting shall be given by the committee at least two weeks
        prior to the meeting.
      \item Notice of any proposed constitutional amendment shall be
        given by the committee at least two weeks prior to the general
        meeting.
      \item A general meeting shall be chaired by a committee
        member. The minutes of the meeting shall be made available
        after the meeting.
      \item At a general meeting resulting in the dissolution of the
        Hacklab, the remaining assets and property of the Hacklab
        shall be used to cover any outstanding expenses of the Hacklab
        and/or be distributed amongst the full members.
      \item At a general meeting, a vote shall be carried if more than
        half of the full members present are in favour.
      \item At a general meeting, an amendment to the constitution
        shall be passed if more than two thirds of the members present
        are in favour.
      \end{enumerate}
\end{enumerate}
\end{document}
