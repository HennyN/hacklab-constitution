% Edinburgh Hacklab constitution
% Drafted by GDE
\documentclass{article}
\usepackage{palatino}
\title{Constitution of Edinburgh Hacklab SCIO - DRAFT}
\date{3rd November 2011}
% Following method of clause numbering is outlined in comp.text.tex
% posting of 16th Feb 2009
% 
% Define a macro to allow the hanging of text from a marker
\makeatletter
\newcommand*{\hangfrom}[1]{%
  \setbox\@tempboxa\hbox{{#1}}%
  \hangindent \wd\@tempboxa
  \noindent\box\@tempboxa}
\makeatother
% Define clause numbering counters
\newcounter{clause}
\renewcommand\theclause{\arabic{clause}}
\newcounter{subclause}[clause]
\renewcommand\thesubclause{\arabic{clause}.\arabic{subclause}}
\newcounter{subsubclause}[subclause]
\renewcommand\thesubsubclause{\arabic{clause}.\arabic{subclause}.\arabic{subsubclause}}
\newcommand\clause[1]{%
  \refstepcounter{clause}
  \par\hangfrom{{\csname theclause\endcsname}\quad }{#1\par}}
\newcommand\subclause[1]{%
  \refstepcounter{subclause}
  \par\hangfrom{{\csname thesubclause\endcsname}\quad }{#1\par}}
\newcommand\subsubclause[1]{%
  \refstepcounter{subsubclause}
  \par\hangfrom{{\csname thesubsubclause\endcsname}\quad }{#1\par}}
% 
% Shortcut to save some typing
\newcommand{\charityact}{Charities and Trustee Investment (Scotland) Act 2005}
% End of preamble
\begin{document}
\maketitle
\section{General}
\subsection{Type of organisation}
\clause{The organisation shall, upon registration, be a Scottish
Charitable Incorporated Organisation (SCIO)}
\subsection{Name}
\clause{The name of the organisation shall be ``Edinburgh Hacklab
SCIO'', hereafter referred to as ``the Hacklab''.}
\subsection{Location}
\clause{The principal office of the Hacklab shall be in Scotland (and
must remain in Scotland).}
\clause{The Hacklab shall hold its regular activities in Edinburgh.}
\subsection{Purposes}
\clause{The Hacklab's purposes are\footnote{GDE: I've changed
  this from ``The main purpose of the Hacklab is to...'' because I'm
  concerned about the application of the charity test, specifically
  around the area of private benefit; we're allowed some private
  benefit to members but I don't think it can be the primary purpose
  of the lab.}:}
\begin{itemize}
\item promote and and encourage technical, scientific and artistic
  skills and innovation through individual projects, collaboration and
  education.
\item provide open events to allow the wider community to meet and
  socialise together.
\item promote and support the use and development of free and open
  technologies, standards, ideas, hardware and software for the
  benefit of all.
\item to provide a shared physical workspace, tools, storage and other
  resources for its members.
\item work with other bodies with similar or complementary objectives.
\end{itemize}
\subsection{Powers}

\clause{The Hacklab has power to do anything which is calculated to
further its purposes or is conducive or incidental to doing so.
\clause No part of the income or property of the Hacklab may be paid
or transferred (directly or indirectly) to the members - either in the
course of the Hacklab's existence or on dissolution - except where
this is done in direct furtherance of the Hacklab's charitable
purposes.}

\subsection{Liability of Members}
\clause{\label{clause:zero_liability} The members of the Hacklab have
  no liability to pay any sums to help meet the debts (or other
  liabilities) of the Hacklab if it is wound up; accordingly, if the
  Hacklab is unable to meet its debts, the members will not be held
  responsible.}

\clause{The members and charity trustees have certain
  legal duties under the \charityact; and clause
  \ref{clause:zero_liability} above does not exclude (or limit) any
  personal liabilities they might incur if they are in breach of those
  duties or in breach of other legal obligations or duties that apply
  to them personally}

\subsection{Use of Resources}

\clause{The Hacklab shall make no claim and take no responsibility for
the projects created by users of the Hacklab resources.}

\clause{Use of Hacklab facilities and equipment shall be at the user's
own risk.}

\clause{The Hacklab shall not be held responsible nor liable for
any actions or behaviour of individuals or groups, whether
members or guests.}

\subsection{General Structure}
\clause{The structure of the Hacklab consists of}

  \subclause{the MEMBERS - who have the right to attend members'
  meetings (including any annual general meetin) and have important
  powers under the constitution; in particular, the members appoint
  people to serve on the board and take decisions on changes to the
  constitution itself;}

  \subclause{the BOARD - who hold regular meetings, and generally
  control the activities of the organisation; for example; the board
  is responsible for monitoring and controlling the financial position
  of the organisation}

\clause{The people serving on the board are referred to in this
constitution as CHARITY TRUSTEES.}
\section{MEMBERS}

\subsection{General}

\clause{A member shall be permitted to attend any physical workspaces
at any time, to make use of all Hacklab resources and to accompany
guests\footnote{GDE: I've taken the ``key'' right out of this due to
  our current difficulties}.}

\subsection{Qualifications for membership}

\clause{Membership is open to any individual aged 18 or over who
[insert membership qualifications]\footnote{GDE: totally not sure what
  we want to put in here. How should we define membership eligibility?
  Living in Edinburgh? Living in East of Scotland? There's a pretty
  full discussion in the OSCR Charity Test document}}

\clause{Employees of the organisation are not eligible for membership.}

\subsection{Application for membership}

\clause{Any person who wishes to become a member must submit a written
application for membership, along with a remittance to meet the first
month's subscription; the application will then be considered by the
board at its next board meeting.}

\clause{The board may, at its discretion, refuse to admit any person
to membership.\footnote{GDE: The SCVO Model Constitution that I'm
  working from doesn't say all members can veto an application, just
  the board. If anyone can find counter-information to this in the
  OSCR documentation or the primary legislation, please let me
  know. Tempted to draft something more like our current constitution
  and submit to see what happens. Thoughts?}}
 
\clause{The board must notify each applicant promptly (in writing or
by email) of its decision on whether or not to admit him/her to
membership.}

\subsection{Membership subscription}

\clause{Members shall require to pay a monthly membership
subscription; unless and until otherwise determined by the members,
the amount of the monthly membership subscription shall be \pounds 30.}

\clause{The monthly membership subscriptions shall be payable on or
before the day of the month on which the member originally joined in
each month.}

\clause{The members may vary the amount of the monthly membership fee
by way of a resolution to that effect passed at an AGM.}

\clause{If the membership subscription payable by any member remains
  outstanding more than 3 months\footnote{GDE: I've been working on
    the basis of not chasing subs until they are 3 months overdue - do
    we want to tighten this up?} after the date on which it fell due -
  and providing he/she has been given at least one written reminder -
  the board may, by resolution to that effect, expel him/her from
  membership.}

\clause{A person who ceases (for whatever reason) to be a member shall
not be entitled to any refund of the membership subscription.}

\subsection{Register of members}

\clause{The board must keep a register of members, setting out:}

\subclause{for each current member:}

\subsubclause{his/her full name and address; and}

\subsubclause{the date on which he/she was registered as a member of
the organisation}

\subclause{for each former member - for at least 6 years from the date
on he/she ceased to be a member:}

\subsubclause{his/her name; and}

\subsubclause{the date on which he/she ceased to me a member.}

\clause{The board must ensure that the register of members is updated
within 28 days of any change:}

\subclause{which arises from a resolution of the board or a resolution
passed by the members of the Hacklab; or}

\subclause {which is notified to the Hacklab.}

\clause{If a member or charity trustee of the Hacklab requests a copy
of the register of members, the board must ensure that a copy is
supplied to him/her within 28 days, providing the request is
reasonable; if the request is made by a member (rather than a charity
trustee), the board may provide a copy which has the addresses blanked
out.}

\subsection{Withdrawal from membership}

\clause{Any person who wants to withdraw from membership must give
written notice of withdrawal to the Hacklab, signed by him or her; he
or she will cease to be a member as from the time when the notice is
received by the Hacklab.}

\clause{Upon ceasing to be a members, all keys, access tokens or
Hacklab property must be returned to the Hacklab.}

\subsection{Transfer of  membership}

\clause{Membership of the Hacklab may not be transferred by a member.}

\subsection{Expulsion from membership}

\clause{Any person may be expelled from membership by way of a
resolution passed by not less than two thirds of those present and
voting at a members' meeting, providing the following procedures have
been observed:}

\subclause{at least 21 days notice of the intention to propose the
resolution must be given to the member concerned, specifying the
grounds for the proposed expulsion;}

\subclause{the member concerned will be entitled to be heard on the
resolution at the members' meeting at which the resolution is
proposed.}

\section {DECISION MAKING BY THE MEMBERS}

\subsection{Members' Meetings}

\clause{The board must arrange a meeting of members (an annual general
meeting or ``AGM'' in the month of April each calendar year.}

\clause{The business of each AGM must include:}

\subclause{a report by the chair on the activities of the organisation;}

\subclause{consideration of the annual accounts of the organisation; and}

\subclause{the election/re-election of charity trustees, as referred
to in clauses \ref{clause:58} to \ref{clause:61}.}

\clause{The board may arrange an extraordinary general meeting at any time.}

\subsection{Power to request the board to arrange a extraordinary
  general meeting (EGM)}

\clause{\label{clause:egm}The board must arrange an extraordinary
general meeting (EGM) if they are requested to do so by a notice
(which may take the form of two or more documents in the same terms,
each signed by one or more members) by members who amount to
5\percent\ or more of the total membership of the organisation at the
time, providing:}

\subclause{the notice states the purposes for which the meeting is to
be held; and}

\subclause{those purposes are not inconsistent with the terms of this
constitution, the \charityact\ or any
other statutory provision.}

\clause{If the board receives a notice under clause \ref{clause:egm},
the date for the meeting which they arrange in accordance with the
notice must not be later than 28 days from the date on which they
received the notice.}

\subsection{Notice of general meetings}

\clause{\label{clause:meeting_notice}At least 14 clear days notice must
be given of any AGM or any EGM.}

\clause{The notice calling the meeting must specify in general terms
what business is to be dealt with at the meeting; and}

\subclause{in the case of a resolution to alter the constitution, must
set out the exact terms of the proposed alteration(s); or}

\subclause{in the case of any other resolution falling within clause
\ref{clause:45} (requirement for a two-thirds majority) must set out
the exact terms of the resolution.}

\clause{The reference to ``clear days'' in clause
\ref{clause:meeting_notice} shall be taken to mean that, in
calculating the days of notice,}

\subclause{the day after the notices are posted (or sent by email)
should be excluded; and

\subclause{the day of the meeting itself should also be excluded.}

\clause{Notice of every AGM and EGM must be given to all the members
of the Hacklab, and to all the charity trustees; but the accidental
omission to give notice to one or more members will not invalidate the
proceedings at the meeting.}

\clause{Any notice which requires to be given to a member under this
constitution must be sent by e-mail to the member, at the e-mail
address last notified by him/her to the Hacklab.}

\subsection{Procedure at AGMs and EGMs}

\clause{No valid decisions can be taken at any AGM or EGM unless a
quorum is present.}

\clause{The quorum for a general meeting is one half the membership at
the time of the meeting, present in person\footnote{GDE: this is a
  guess. 1/3 seems too few.}.}

\clause{If quorum is not present within 15 minutes after the time at
which a general meeting was due to start - or if a quorum ceases to be
present during a general meeting - the meeting cannot proceed; and
fresh notices of meeting will require to be sent out, to deal with the
business (or remaining business) which was intended to be conducted.}

\clause{A charity trustee should act as the chairperson of each general
meeting.}

\subsection{Voting at general meetings}

\clause{Every member has one vote, which must be given personally.}

\clause{All decisions at general meetings will be made by majority
vote - with the exception of the types of resolution listed in clause
\ref{clause:supermajority}.}

\claus{The following resolutions will be valid only if passed by not
less that two-thirds of those voting on the resolution at an AGM or
EGM (or if passed by way of a written resolution under clause
\ref{clause:written_resolution}:}

\subclause{a resolution amending the constitution;}

\subclause{a resolution expelling a person from membership under
clause \ref{clause:expel};}

\subclause{a resolution directing the board to take any particular
step (or directing the board not to take any particular step);}

\subclause{a resolution approving the amalgamation of the Hacklab with
another SCIO (or approving the constitution of the new SCIO to be
constituted as the successor pursuant to that amalgamation);}

\subclause{a resolution to the effect that all of the Hacklab's
property, rights and liabilities should be transferred to another SCIO
(or agreeing to the transfer from another SCIO of all of its property,
rights and liablities);}

\subclause{a resolution for the winding up or dissolution of the
organisation.}

\clause{If there is an equal number of votes for and against any
resolution, the chairperson of the meeting will be entitled to a
second (casting) vote.}

\clause{A resolution put to the vote at a general meeting will be
decided on a show of hands - unless the chair (or at least two other
members present at the meeting) ask for a secret ballot.}

\clause{The chairperson will decide how any secret ballot is to be
conducted, and he/she will declare the result of the ballot at the
meeting\footnote{GDE: this allows us, in particular, to elect trustees
  without having to state explicitly, how the election is going to happen}.}

\subsection{Written resolutions by members}

\clause{A resolution agreed to in writing (or by e-mail) by all the
members will be as valid as if it had been passed at a general
meeting; the date of the resolution will be taken to be the date on
which the last member agreed to it.}

\subsection{Minutes}

\clause{The board must ensure that proper minutes are kept in relation
to all general meetings.}

\clause{Minutes of general meetings must include the names of those
present; and (so far as possible) should be signed by the chairperson
of the meeting.}

\section{Board}

\subsection{Number of charity trustees}

\clause{There shall be 3 charity trustees\footnote{GDE: I'm starting
    to think that we need 5 trustees (with a quorum of 3) to make sure
    we are legal at all times.}}

\subsection{Treasurer}

\clause{One of the trustees shall be the treasurer\footnote{GDE: The
  SVCO model constitution has a chair, secretary and treasurer from
  the board. I know that secretary is not a legal SCIO requirement but
  I'm not sure about the position of chair - given our desire to
  remain a ``flat'' organisation this is one I think we submit and
  modify if it doesn't fly with OSCR.}.}

\clause{A person elected to be treasurer will automatically cease to
hold that office:}

\subclause{if he/she ceases to be a charity trustee; or}

\subclause{if he/she gives to the organisation a notice of resignation
from that office, signed by him/her.}

\subsection{Eligibility}

\clause{A person will not be eligible for election to the board unless
he/she is a member of the Hacklab.}

\clause{A person will not be eligible for election to the board if
he/she is:}

\subclause{disqualified from being a charity trustee under the
\charityact; or}

\subclause{an employee of the Hacklab.}

\subsection{Initial charity trustees}

\clause{The individuals who signed the charity trustee declaration
forms which accompanied the application for incorporation of the
Hacklab shall be deemed to have been appointed by the members as
charity trustees with effect from the date of incorporation of the
organisation.}

\subsection{Election, retiral, re-election}

\clause{\label{clause:boardelection}At each AGM, the members may elect
any member (unless he/she is debarred from membership under clause
\ref{clause:disqualified}) to be the treasurer and the other two
charity trustees.}

\clause{At each AGM, all of the charity trustees must retire from
office - but shall then (subject to clause \ref{clause:maxperiod}) be
eligible for re-election under clause \ref{clause:boardelection}.}

\clause{\label{clause:maxperiod}A person who has served on the board
for a period of 3 years shall automatically vacate office on expiry of
that 3 year period and shall not be eligible for re-election until a
further year has elapsed.}

\clause{For the purposes of clause \ref{clause:maxperiod}:}

\subclause{the period from the date of formation of the Hacklab to the
first AGM shall be deemed to be a period of one year, unless it is of
less than 6 months duration (in which case it shall be disregarded);}

\subclause{the period between the date of election of a charity
trustee and the AGM which next follows shall be deemed to be a period
of one year, unless it is of less than six months duration in which
case it shall be disregarded;}

\subclause{the period between one AGM and the next shall be deemed to
be a period of one year;}

\subclause{if a charity trustee ceases to hold office but is
reelected to that office within a period of six months, he/she shall
be deemed to have held office as a charity trustee continuously.}

\clause{A charity trustee retiring at an AGM will be deemed to have
been re-elected unless:}

\subclause{he/she advises the board prior to the conclusion of the AGM
that he/she does not wish to be reappointed as a charity trustee; or}

\subclause{an election process was held at the AGM and he/she was not
among those elected/re-elected through that process.}

\subsection{Termination of office}

\clause{A charity trustee will automatically cease to hold office if:}

\subclause{he/she becomes disqualified from being a charity trustee
under the \charityact;}

\subclause{he/she becomes incapable for medical reasons of carrying
out his/her duties as a charity trustee - but only if that has
continued (or is expected to continue) for a period of more than six
months;}

\subclause{he/she ceases to be a member of the Hacklab;}

\subclause{he/she becomes an employee of the Hacklab;}

\subclause{he/she gives the Hacklab a notice of resignation, signed by
him/her;}

\subclause{he/she is absent (without good reason, in the opinion of
the board) from more than three consecutive meetings of the board -
but only if the board resolves to remove him/her from office.}

\subclause{\label{clause:materialbreach}he/she is removed from office
  by resolution of the board on the grounds that he/she is considered
  to have committed a material breach of the code of conduct for
  charity trustees (as referred to in clause
  \ref{clause:codeofconduct}).}

\subclause{\label{clause:charityactbreach}he/she is removed from office
by resolution of the board on the grounds that he/she is considered to
have been in serious or persistent breach of his/her duties under
section 66(1) or (2) of the \charityact; or}

\subclause{\label{clause:removedbymembers}he/she is removed from
  office by a resolution of the members passed at a general meeting.}

\clause{A resolution under paragraph \ref{clause:materialbreach},
\ref{clause:charityactbreach}, or \ref{clause:removedbymembers} shall
be valid only if:}

\subclause{the charity trustee who is the subject of the resolution is
given reasonable prior written notice of the grounds upon which the
resolution for his/her removal is to be proposed;}

\subclause{the charity trustee concerned is given the opportunity to
address the meeting at which the resolution is proposed, prior to the
resolution being put to the vote; and}

\subclause{(in case of a resolution under clause
\ref{clause:materialbreach} or \ref{clause:charityactbreach}) at least
two thirds (to the nearest round number) of the charity trustees then
in office vote in favour of the resolution}

\subsection{Register of charity trustees}

\clause{The board must keep a register of charity trustees, setting
out:}

\subclause{for each current charity trustee;}

\subsubclause{his/her full name and address;}

\subsubclause{the date on which he/she was appointed as a charity
trustee; and}

\subsubclause{whether he/she is the treasurer.}

\subclause{for each former charity trustee - for at least 6 years from
the date on which he/she ceased to be a charity trustee:}

\subsubclause{the name of the charity trustee;}

\subsubclause{whether he/she was the treasurer; and}

\subsubclause{the date on which he/she ceased to be a charity trustee.}

\clause{The board must ensure that the register of charity trustees is
updated within 28 days of any change:}

\subclause{which arises from a resolution of the board or a resolution
passed by members of the Hacklab; or}

\subclause{which is notified to the Hacklab.}

\clause{If any person requests a copy of the register of charity
trustees, the board must ensure that a copy is supplied to him/her
within 28 days, providing the request is reasonable; if the request is
made by a person who is not a charity trustee of the Hacklab, the
baord may provide a copy which has the addresses blanked out - if the
SCIO is satisfied that including that information is likely to
jeopardise the safety or security of any person or premises.}

\subsection{Powers of board}

\clause{Except where this constitution states otherwise, the
Hacklab (and its assets and operations) will be managed by the
board; and the board may exercise all the powers of the Hacklab.}

\clause{A meeting of the board at which a quorum is preset may
exercise all powers exercisable by the board.}

\clause{The members may, by way of a resolution passed in compliance
with clause \ref{clause:supermajority} (requirement for two-thirds
majority), direct the board to take any particular step or direct the
board not to take any particular step; and the board shall give effect
to any such direction accordingly.}

\subsection{Charity trustees - general duties}

\clause{\label{clause:trusteeduties}Each of the charity trustees has a
  duty, in exercising functions as a charity trustee, to act in the
  interests of the Hacklab; and in particular, must:}

\subclause{seek, in good faith, to ensure that the organisation acts
  in a manner which is in accordance with its purposes;}

\subclause{act with the care and diligence which it is reasonable to
expect of a person who is managing the affairs of another person;}

\subclause{in circumstances giving rise to the possibility of a
conflict of interest between the Hacklab and any other party:}

\subsubclause{put the interests of the Hacklab before that of the
other party;}

\subsubclause{where any other duty prevents him/her from doing so,
disclose the conflicting interest to the Hacklab and refrain from
participating in any deliberation or decision of the other charity
trustees with regard to the matter in question;}

\subclause{ensure that the Hacklab complies with any direction,
requirement, notice or duty imposed under or by virtue of the
\charityact.}

\clause{In addition to the duties outlined in clause
  \ref{clause:trusteeduties}, all of the charity trustees must take
  such steps as are reasonably practicable for the purpose of
  ensuring:}

\subclause{that any breach of any of those duties by a charity trustee
  is corrected by the charity trustee concerned and not repeated; and}

\subclause{that any trustee who has been in serious and persistent
  breach of those duties is removed as a trustee.}

\clause{Provided he/she has declared his/her interest - and has not
  voted on the question of whether or not the Hacklab should enter
  into the arrangement - a charity trustee will not be debarred from
  entering into an arrangment with the Hacklab in which he/she has a
  personal interest; and (subject to clause
  \ref{clause:trusteeasemployee} and to the provisions relating to
  remuneration for services contained in the \charityact), he/she may
  retain any personal benefit which arises from that arrangement.}

\clause{\label{clause:trusteeasemployee}No charity trustee may serve as
an employee (full time or part time) of the Hacklab; and no
charity trustee may be given any remuneration by the Hacklab for
carrying out his/her duties as a charity trustee.}

\clause{The charity trustees may be paid all travelling and other
expenses reasonably incurred by them in connection with carrying out
their duties; this may include expenses relating to their attendance
at meetings\footnote{GDE: this is boilerplate from the model
  consitution and is standard practice for other charities I've been
  directly or indirectly involved with, where the trustees may be
  geographically scattered. Do we feel we need this clause?}.}

\subsection{Code of conduct for charity trustees}

\clause{Each of the charity trustees shall comply with the code of
  conduct (incorporating detailed rules on conflict of interest)
  prescribed by the board from time to time\footnote{GDE: no, I don't
    have a code of conduct written. The commentary on this clause says
    ``The reference to a code of conduct is in line with principles of
    best practice in governance.'' If anyone can decode that into what
    the code of conduct should actually look like, please jump in.}.}

\clause{The code of conduct referred to in clause
  \ref{clause:codeofconduct} shall be supplemental to the conduct of
  charity trustees contained in the constitution and the duties
  imposed on charity trustees under the \charityact; and all relevant
  provisions of this consitution shall be interpreted and applied in
  accordance with the provisions of the code of conduct in force from
  time to time.}

\section{DECISION-MAKING BY THE CHARITY TRUSTEES}

\subsection{Notice of board meetings}

\clause{Any charity trustee may call a meeting of the board.}

\clause{At least 7 days' notice must be given of each board meeting,
  unless (in the opinion of the person calling the meeting) there is a
  degree of urgency which makes that inappropriate.}

\subsection{Procedure at board meetings}

\clause{\label{clause:boardquorum}No valid decisions can be taken at a
  board meeting unless a quorum is present; the quorum for board
  meetings is 2 charity trustees, present in person.}

\clause{If at any time the number of charity trustees in office falls
  below the number stated as the quorum in clause
  \ref{clause:boardquorum}, the remaining charity trustees will have
  power to call an EGM - but will not be able to take any other valid
  decisions.}

\clause{At each board meeting, the charity trustees present must elect
  (from amonsgt themselves) the person who will act as chairperson of
  that meeting.}

\clause{Each charity trustee has one vote, which must be given in
  person.}

\clause{All decisions at board meetings will be made by majority
  vote.}

\clause{If there is an equal number of votes for and against any
  resolution, the chairperson of the meeting will be entitled to a
  second (casting) vote.}

\clause{The board may, at its discretion, allow any person to attend
  and speak at a board meeting notwithstanding that he/she is not a
  charity trustee - but on the basis that he/she must not participate
  in decision-making.}

\clause{\label{clause:conflictofinterest}A charity trustee must not
  vote at a board meeting (or at a meeting of a sub-committee) on any
  resolution which relates to a matter in which he/she has a personal
  interest or duty which conflicys (or may conflict) with the
  interests of the Hacklab; he/she must withdraw from the meeting
  while an item of that nature is being dealt with.}

\clause{For the purposes of clause \ref{clause:conflictofinterest}:}

\subclause{an interest held by an individual who is ``connected'' with
  the charity trustee under section 68(2) of the \charityact
  (husband/wife, partner, child, parent, brother/sister etc) shall be
  deemed to be held by that charity trustee.}

\subclause{a charity trustee will be deemed to have a personal
  interest in relation to a particular matter if a body in relation to
  which he/she is an employee, director, member of the management
  committee, officer or elected representative has an interest in that
  matter.}

\subsection{Minutes}

\clause{\label{clause:properminutes}The board must ensure that proper
  minutes are kept in relation to all board meetings and meetings of
  sub-committees.}

\clause{The minutes to be kept under clause \ref{clause:properminutes}
  must include the names of those present; and (so far as possible)
  should be signed by the chairperson of the meeting.}

\section{Administration}

\subsection{Delegation to sub-committees}

\clause{\label{clause:delegation}The board may delegate any of their
  powers to sub-committees; a sub-committee must include at least one
  charity trustee, but other members of a sub-committee need not be
  charity trustees.}

\clause{When delegating powers under clause \ref{clause:delegation},
  the board must set out appropriate conditions (which must include an
  obligation to report regularly to the board).}

\clause{Any delegation of powers under clause \ref{clause:delegation}
  may be revoked or altered by the board at any time.}

\clause{The rules of procedure for each sub-committee, and the
  provisions relating to membership of each sub-committee, shall be
  set by the board.}

\subsection{Operation of accounts}

\clause{Three account signatories will be appointed by the board;
  these must themselves all be charity trustees.}

\clause{\label{clause:signatories}Subject to clause
  \ref{clause:electronicbanking}, the signatures of two out of three
  signatories will be required in relation to all operations (other
  than the lodging of funds) on the bank and building society accounts
  held by the Hacklab.}

\clause{\label{clause:electronicbanking}Where the Hacklab uses
electronic facilities for the operation of any bank or building
society account, the authorisations required for operations on that
account must be consistent with the approach reflected in clause
\ref{clause:signatories}\footnote{GDE: it will be interesting to see
  if we can set this up on the bank account.}.}

\subsection{Accounting records and annual accounts}

\clause{The board must ensure that proper accounting records are kept,
  in accordance with all applicable statutory requirements.}

\clause{The board must prepare annual accounts, complying with all
  relevant statutory requirements; if an audit is required under any
  statutory provisions (or if the board consider that an audit would
  be appropriate for some other reason), the board should ensure that
  an audit of the accounts is carried out by a qualified
  auditor\footnote{GDE: I don't know if we need to be audited or not,
    I've not read the guidance yet.}}

\clause{The financial year shall run from 1st April to 31st March.}

\section{Miscellaneous}

\subsection{Winding-up}

\clause{If the Hacklab is to be wound up or dissolved, the winding-up
or dissolution process will be carried out in accordance with the
procedures set out under the \charityact.}

\clause{Any surplus assets available to the organisation immediately
preceding its winding up or dissolution must be used for purposes
which are the same as - or which closely resemble - the purposes of
the organisation as set out in this constitution.}

\subsection{Alterations to the constitution}

\clause{This constitution may (subject to clause
\ref{clause:oscrconsent}) be altered by resolution of the mmebers
passed at an AGM or EGM (subject to achieving the two-thirds majority
referred to in clause \ref{clause:supermajority}) or by way of a
written resolution of the members.}

\clause{\label{clause:oscrconsent}The \charityact\ prohibits taking
certain steps (e.g. change of name, an alteration to the purposes,
amalgamation, winding-up) without the consent of the Office of the
Scottish Charity Regulator (OSCR).}

\subsection{Interpretation}

\clause{References in this constitution to the \charityact\ should be
taken to include:}

\subclause{any statutory provision which adds to, modifies or replaces
that Act; and}

\subclause{any statutory instrument issued in pursuance of that Act or
in pursuance of any statutory provision falling under clause
\ref{clause:statutoryprovisions} above.}

\clause{In this constitution:}

\subclause{``Charity'' means a body which is either a ``Scottish
charity'' within the meaning of section 13 of the \charityact\ or a
``charity'' within the meaning of section 1 of the Charities Act 2006,
providing (in either case) that its objects are limited to charitable
purposes;}

\subclause{``charitable purpose'' means a charitable purpose under
section 7 of the \charityact\ which is also regarded as a charitable
purpose in relation to the application of the Taxes act.}

\end{document}
